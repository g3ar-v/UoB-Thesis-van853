\chapter{Conclusions}
The dissertation presented a new solution for attendance monitoring using other perspectives to address the inefficiencies prevalent in existing systems. It also fills the gap found in the literature on monitoring systems. 

While reviewing the literature and noting their central argument, it was observed that most of the literature sought to tackle attendance with one solution. Which is to improve the efficiency of the attendance method used. Though there are major solutions that address the argument, they seem to solve this argument in one perspective. The perspective that is advanced is one that enhances the efficiency of these methods against the background of the circumstances it is monitoring. The project is thus a combination of the methods stated in some of the literatures and a novel method of enhancing their efficiency. The project aimed to monitor attendance with multiple methods and also handle unreliable network lapses from the database or the internet. 

In view of the above, a system that monitors attendance with a web application to interact with the process was designed. A manual method website that enables the administrator to click students present was incorporated. While a \gls{RFID} method and a fingerprint method was also used. Each component of the system communicates with each other via a network. This offers an advantage by enabling the administrator to remotely access the web application from anywhere.

A dummy dataset was therefore used to test the system. The dataset was used in the database and the database was used to enrol the \gls{RFID} tags and fingerprints. The project was able to monitor attendance for each event, monitoring attendance with the \gls{RFID}, fingerprint or manually on the website. The system worked perfectly as tracking based on entry or entry and exit was achieved. The system is therefore recommended for adaptation for use by the University of Birmingham.

As much as the project answers the research questions and its aims, there were limitations due to the constraints of time and resources. The major limitations include that posed by the implementation of an authentication page. As stated in the Background\textsuperscript{\ref{authentication}}, a web application needs authentication to supply the resources to the right person. The system is not designed to function asynchronously for seperate lecturers. In a real world scenario, there are multiple administrators of this system and they will need to monitor attendance separately. An option to run some of these methods concurrently would improve the speed and flexibility of this system.

Nonetheless, the next step would be to create a mobile application that tracks attendance using a QR code. The website will display a generated QR code when the QR code method button is clicked. The mobile application checks the location of students before scanning the QR code. The mobile application also has a virtual card that functions as an NFC tag to track attendance. This is a way to provide security and an improvement from the RFID tag which is susceptible to attendance by proxy. The next step would be to make all the designed methods run concurrently so as to make it more flexible. The analytics page would have a list of the dates that attendance has been monitored and from this list one would be able to see the attendance status of that event. The login/signup page would be created and authenticated using the lecturer table in the database.


