
% book - Default class for a normal book
\documentclass[a4paper,12pt,oneside]{book}

%textbook - a class for text bigger than 12pt
%\documentclass[a4paper,14pt,twoside,openright,reqno,table]{extbook}

% Options in detail:
% openany - allows chapter and similar openings to occur on left hand pages
% openright - allows chapter and similar openings tohttps://www.overleaf.com/project/620273a89e158f6264034920 occur on right hand pages
% fleqn  - left-alignment of formulas
% leqno - labels formulas on the left-hand side instead of right
% reqno - labels formulas on the right-hand side
% draft - in draft mode the figures are not loaded, useful for speeding up typesetting
% onecolumn or twocolumn
% oneside (default for article and report)
% twoside (default for book)
% table  --> to avoid the message: package xcolor has already been loaded ...
\usepackage{packages}
\usepackage{csquotes}
% Almost all the settings are defined in packages.sty

% Put a grey textual watermark on document pages (PS mode only)
%\usepackage[italian,light,first,bottomafter]{draftcopy}

% Put a grey textual watermark on document pages (PDF mode)
%\usepackage{draftwatermark}
% If you want to change the default DRAFT text
%\SetWatermarkText{DRAFT}
% If you want to change the default grey color of the text
%\SetWatermarkColor{red}

%%%%%%%%%%%%%%%%%%%%%%%%%%%%%%%%%%%%%%%%%%%
%   DOCUMENT: an ordered list of files    %
%             that you can include or not %
%             in your document            %
%%%%%%%%%%%%%%%%%%%%%%%%%%%%%%%%%%%%%%%%%%%
\begin{document}

% Title Page %
% ====================== Centred Title Page ===========================

% \begin{titlepage}
% \thispagestyle{empty}
% \centering
% %\providecommand\pdfbookmark[3][]{} \pdfbookmark[0]{Title Page}{bm:Title}
% \vspace*{5cm}
% \textsc{\huge{First Line of Your Title}}\\[0.5em]
% \textsc{\huge{Second Line}}\\[0.5em]
% \textsc{\huge{Third}}\\[0.5em]
% \vfill
% By\\[0.5em]
% \textsc{\Large{Joseph Chamberlain}}
% \vfill
% A thesis submitted to\\[-0.8em]
% the University of Birmingham\\[-0.8em]
% for the  degree of\\[-0.8em]
% \MakeUppercase{Doctor of Philosophy} \\[\baselineskip]
% \begin{figure}[ht!]
% \begin{center}
% \includegraphics[height=4cm]{frontmatter/images/BirminghamUniversityCrest.png}
% \end{center}
% \end{figure}
% Cold Atoms Research Group\\[-0.8em]
% School of Physics and Astronomy\\[-0.8em]
% College of Engineering and Physical Sciences\\[-0.8em]
% University of Birmingham\\[-0.8em]
% September~2020 \\[\baselineskip]
% %\copyright\ Copyright by \MakeUppercase{\@Author},~\@Year\\
% %All Rights Reserved
% \end{titlepage}
% \clearpage

% ================ Aligned Title Page ===========================

\thispagestyle{empty}
\providecommand\pdfbookmark[3][]{} \pdfbookmark[0]{Title Page}{bm:Title}
\vspace*{1cm}
\begin{figure}[ht!]
  \includegraphics[width=6.5cm]{frontmatter/images/full-colour-logo.eps}
\end{figure}
\vfill
\begin{flushleft}
  \textsc{\huge{Tackling internet inefficiency }}\\[0.5em]
  \textsc{\huge{in modern-day}}\\[0.5em]
  \textsc{\huge{Attendance monitoring Systems}}\\[0.5em]
  \vfill
  By\\[\baselineskip]
  \textsc{\Large{Victor Akaninyene Nyoyoko}}
  \vfill
  1961453\\[-0.8em]
  MEng Computer Science / Software Engineering FT\\[-0.8em]
  \vfill
  supervised by\\[-0.8em]
  \vfill
  \MakeUppercase{AHMAD IBRAHIM} \\[\baselineskip]
\end{flushleft}
\begin{flushright}
  School of Computer Science\\[-0.8em]
  College of Engineering and Physical Sciences\\[-0.8em]
  University of Birmingham\\[-0.8em]
  September~2021 \\[\baselineskip]
\end{flushright}
% \copyright\ Copyright by \MakeUppercase{\@Nyoyoko Victor},~\@2022\
\vspace*{4cm}
11,014 words.
%All Rights Reserved
\clearpage
%your name
%studentID
%programme name
% supervisor's name
% word count

%% FRONTMATTER %%
% The pages inside of frontmatter are in Roman numerals and the chapters will not have numeration
\frontmatter

% COPYRIGHT %
%% ======================== Copyright page ===============================

\thispagestyle{empty}
%\providecommand\pdfbookmark[3][]{} \pdfbookmark[0]{Copyright}{bm:Copyright}
\addtocounter{page}{-1}
\vspace*{\fill}
\vfill
\begin{center}
\copyright\ Copyright by \MakeUppercase{Joseph Chamberlain},~2020\\
All Rights Reserved
\end{center}
%\vspace{1in}
\clearpage

% ABSTRACT %
\providecommand\phantomsection{} \phantomsection
%\addcontentsline{toc}{part}{Abstract}
\begin{center}
%\hrule
\providecommand\pdfbookmark[3][]{} \pdfbookmark[0]{Abstract}{bm:Abstract}
\vspace*{1in}
\textbf{ABSTRACT}\\[2\baselineskip]
% \vspace*{.1in}
\end{center}

For over four years in the University of Birmingham campus, I  have taken note of the attendance monitoring system, the inefficiency in tracking attendance  in a Wide Area Network and it's causes.There are more places like this with this inability to have a reliable internet connection and have issues with tracking attendance in monitoring system.


I decided to develop an attendance system with a raspberry pi, a website, RFID (Radio Frequency Identification Devices) and a Fingerprint module with Azure Cloud Computing Service where I designed and implemented a database exception handler for this system 

This produced a system that is capable of tracking attendance in cases of short internet latency gaps given different options and tracking choices. It seems to curb the problems of not connecting to database by storing data temporarily on the raspberry-pi and sending it back up when the internet is on.



This has lead to (219 words)

Being tested with large set of data this seems to fix 



% DEDICATION %
% % ======================= Dedication Page ============================
\newpage
\thispagestyle{plain}%
\begin{center}
%\providecommand\pdfbookmark[3][]{}\pdfbookmark[0]{DedicationPage}{bm:Dedicate}
\vspace*{1.575in}
\textbf{DEDICATION}\\[2\baselineskip]
Dedicated to my cats
\end{center}%
\vfill
\newpage


% ACKNOWLEDGEMENTS %
% % ========================= Acknowledgments ==============================
\providecommand\phantomsection{} \phantomsection
%\addcontentsline{toc}{part}{Acknowledgments}
\thispagestyle{plain}
\renewcommand{\baselinestretch}{1}\small\normalsize
\begin{center}
%\providecommand\pdfbookmark[3][]{}\pdfbookmark[0]{Acknowledgments}{bm:Acknowledge}
\vspace*{0.375in}
\textbf{ACKNOWLEDGMENTS}\\[3\baselineskip]
\end{center}
\renewcommand{\baselinestretch}{1.66} \small\normalsize%
I acknowledge the people who helped me.
\newpage

% CREDITS %
% \begin{titlepage}

\nonumber
\null \vspace {\stretch{1}}
%	\begin{flushright}
%	\begin{verse}
    \begin{center}
\textit{Worker bees can leave\\
even drones can leave\\
the queen is their slave} \\[5mm]
%	\end{verse}
	Tyler Durden, ''Fight Club''
%	\end{flushright}
    \end{center}
\vspace{\stretch{2}}\null

\end{titlepage}
\cleardoublepage

% CONTENTS %
% To help hyperref to jump to the correct page
\phantomsection
%\addcontentsline{toc}{chapter}{Contents}
\tableofcontents

\renewcommand{\cftpartfont}{\normalfont\bfseries} % \part font in ToC
\renewcommand{\cftchapfont}{\normalfont\bfseries} % \chapter font in ToC
\renewcommand{\cftchappagefont}{\normalfont}
\renewcommand{\cftpartpagefont}{\normalfont}
\renewcommand{\cftpartleader}{\cftdotfill{\cftdotsep}}
\renewcommand{\cftchapleader}{\cftdotfill{\cftdotsep}}
\renewcommand{\cftsecleader}{\cftdotfill{\cftdotsep}}
% \setcounter{tocdepth}{4}
% \setcounter{secnumdepth}{4}

\cftsetindents{chapter}{0in}{0.25in}
\cftsetindents{section}{0.25in}{0.35in}
\cftsetindents{subsection}{.6in}{0.5in}
\cftsetindents{subsubsection}{1.5in}{0.5in}

\setlength{\cftbeforepartskip}{.25em}
\setlength{\cftbeforechapskip}{.25em}
\setlength{\cftbeforesecskip}{-.5em}
\setlength{\cftbeforesubsecskip}{-.5em}

\addtocontents{toc}{~\hfill\textbf{Page}\par}
\thispagestyle{plain}

\clearpage

%   % Make the list of figures
% \listoffigures
% \thispagestyle{plain}
% \clearpage

%  % Make the list of tables
% \listoftables
% \thispagestyle{plain}
% \clearpage
 

 
 
% GLOSSARY %
\cleardoublepage
% To help hyperref to jump to the correct page
\phantomsection
% To add the Glossary in the table of contents
%\addcontentsline{toc}{chapter}{Glossary}
% Prints the glossary
\printglossary

% In order to update the glossary you have to execute:
% \makeindex -s main.ist -t main.alg -o main.acr main.acn
% to insert an item in the document:
% \newglossaryentry{item_label}{name={item}, description={description}}
% if it doesn't appear you have to initialize it:
% \glsadd{item_label}
% or if it is called again in the following text:
% \gls{item_label}

% ACRONYMS %
\cleardoublepage
% To help hyperref to jump to the correct page
\phantomsection
% To add the Index of Symbols in the table of contents
%\addcontentsline{toc}{chapter}{Acronyms}
% Prints the Acronyms
\printglossary[type=\acronymtype,title=Acronyms]

% In order to update the symbols you have to execute:
% makeindex -s main.ist -t main.glg -o main.gls main.glo
% to insert an item in the document::
% \newacronym{item_label}{name={item}, description={description}}
% if it doesn't appear you have to initialize it:
% \glsadd{item_label}
% or if it is called again in the following text:
% \gls{item_label}

%% MAINMATTER %%
% The pages inside of mainmatter are in Arabic numerals and the chapters will have numeration
\mainmatter

%\part{If you want parts}
\newacronym{IoT}{$IoT$}{Internet of Things}
\glsadd{IoT}

\chapter{Introduction} 
The moment mankind found the need to keep track of attendance in different environments, automation, accuracy and reliability have been the principle. According to \citeauthor{HistoryO76:online} \& \citeauthor{TheEvolu41:online}, the use of attendance tracking dates back to the early 1800s during the Industrial Revolution when families moved from rural areas into the cities for factory labour. Due to the unhealthy and hazardous nature of most of these jobs, the government had to set regulations to enhance the working conditions. A method to keep record of hours spent by workers was developed. The mechanical clock created by William Bundy in 1888, aided in the advancement of this system. It was used to record the time a worker entered and left the factory. This was done on time cards. Overtime, smaller electrical clocks replaced the mechanical clocks and improvements were made in the time cards. In the 1990s, a new Time software was developed to replace the mechanical clocks which were less efficient and expensive. Excel sheets were a new method of storing attendance data. Absence management was monitored effectively.\cite{HistoryO76:online, TheEvolu41:online} Monitoring of attendance in organisations has enhanced the satisfaction of both the employer and the employees due to the transparency of the monitoring systems as a tangible contract held by both parties. In educational institutions, these systems assess the level of participation of users, aided in the procurement  of valid research data and the management of attendance records. Though there are Attendance Monitoring Systems with the capacity to efficiently do their jobs, there are places where lack of a reliable internet connectivity as a result of many external factors like latency and packet loss impede the efficiency of these systems to function effectively. This project investigates through the field of \gls{IoT} to find a distinct solution that solves this problem efficiently.
 
\section{Project Aims}
 
The aim of this project is to build an attendance monitoring system that is able to track attendance with a diverse number of methods and to render a novel solution to the argument "Inefficiency in attendance monitoring systems”\label{problem}. The focus will be on integrating this system to use an \gls{RFID} system, a manual website method and a fingerprint for tracking attendance while also monitoring attendance with a short lapse in internet or database connectivity. The website aids an administrator or lecturer in interacting with the system. The evaluation criterias of this system will be based on its ability;
\begin{itemize}
 \item to track attendance with any of the referred methods,
 \item to handle the exceptions of the internet or database connection problem.
 \item to show results based on each attendance class
\end{itemize}
This project also aims to give the administrator remote access to the \gls{RFID} and fingerprint systems from anywhere in the world theoretically based on its design and implementation. The results will be determined by the use of a dummy dataset. \gls{RFID} tags will be enrolled with these datasets and fingerprints of certain individuals.
\section{Related Works}
Most of the literature cited, agree that manual attendance is time-consuming, prone to mistakes and insecure. Nonetheless, a few of the literatures have revealed measures used to solve the problem of inefficiency in attendance monitoring systems. Papers exist which narrate how \gls{RFID} has been used as a solution. \citeauthor{KaziARP} considers it a waste of time monitoring both body temperature and attendance in schools which was a problem caused by the COVID-19 pandemic. They preferred to join these two systems to monitor attendance using \gls{RFID} as well as monitor body status and store the data to be accessed remotely across the globe.\cite{KaziARP} This influenced the merging of multiple attendance methods in this work and also the use of remote access to the website. \citeauthor{Tt2021} proposed a smart attendance system that is appropriate to evade the use of manual attendance system and reduce manual work of teachers and administrators. By building a system that monitors attendance with passive \gls{RFID} and sends the data attained to the Cloud. Futhermore, a GPS system is also used to get the live location of the student.\cite{Tt2021} \citeauthor{Www2012} seek to create a fast, automated, reliable and accurate system that monitors attendance by reading tags in the lecture room using an Ultra-High frequency \gls{RFID} reader and a middleware to sort, convert and relay data to a database server or to the appropriate system.\cite{Www2012} \citeauthor{Bhagat2020} on the other hand, proposes a system that effectively and efficiently monitors attendance with a website and \gls{RFID} reader using an MQTT protocol and a NodeMCU firmware with the aim of eradicating manual attendance and saving cost.\cite{Bhagat2020}
Some of these works also present solution that uses fingerprint as a method of tracking.
\citeauthor{soe2018implementation} proposes fingerprint based biometric attendance system that monitors attendance and prevents proxies(when someone scans attendance for someone else). The data is stored in an SD card and later extracted. The literature also emphasizes the reliability and security of the system to prohibit truancy and develops a solution to stop it. As a tracking method, fingerprint is more reliable, accurate and cost effective based on the fact that every individual has a unique fingerprint. Furthermore, a GSM system is put in place to alert guardians, if a student does not attend. One limitation of this solution is that it is not automated. A teacher or an authorised personnel has to get an SD card from the microcontroller to store attendance in the database.\cite{soe2018implementation} In another study, \citeauthor{Rachna} sought to reduce the entire time taken to capture attendance of students in a class by designing a portable device that is circulated around the classroom for students to mark their attendance. The device is a cloud-based system that relies on a fingerprint scanner, web application to track attendance, it also reports absentees to their guardians using SMS and a mobile application for students to see their attendance record. The fingerprint module uses MQTT as a means of communicating with its Web application. It seems that the author aims to solve attendance by proxy since only people in the room can sign in as present. Although this solves the problem of proxy, it impaires automation as circulating this device among a large number of students might not seem efficient. The authors future remarks hinted at a device that operates in an offline mode which stores the attendance locally when there is no network connectivity and transfers it to the cloud when there is a connection.\cite{Rachna} However, having perused some literatures in this field, this is the only paper that addresses the implementation but failed to implement. It is therefore the intention of this project to implement a system which stores a temporary file directory on the Raspberry pi before being sent to the database. Similarly \citeauthor{Rahman2021}, aims to provide a reliable method of monitoring attendance with the use of a fingerprint automated system. This is with the use of a user interface to interact with both the lecturers and student. The user interface acts as a response system for students but can be used to start attendance with the use of a lecturer's fingerprint.\cite{Rahman2021} Lessons from \citeauthor{Rahman2021}'s work be useful in this work.

There are other methods used to monitor attendance. \citeauthor{9716261} proposes an attendance system using Speech recognition with the aim of impeding the use of manual attendance and increasing efficiency by scanning attendance with the use of voice input on a mobile application. A speech recognizer converts the audio input into a valid message which matches a database record.\cite{9716261}
 
Given the existing projects and literature in this field, most of the authors reviewed, focused on upgrading existing methods to solve the inefficiency of the attendance monitoring system.\cite{KaziARP, Tt2021, Www2012, Bhagat2020, soe2018implementation, Rachna} In as much as efficiency is the central theme in these literatures, it has been noted that solutions based on only one perspective have been applied by these authors. Notwithstanding, efficiency has always been the driving force of these efforts. My argument is, "why not use multiple methods for different scenarios"?, given that in some cases some methods might be more efficient in taking attendance and having them integrated makes the system flexible and better or even more efficient(in theory) in tracking attendance. A paper exists that implements this. \citeauthor{6266137} proposed a system that uses NFC and specifically a fingerprint module as a means to curb truancy, monitor attendance and provide reliable information on student attendance from a component called the back office. However, the central theme of this paper has to do with monitoring attendance in general and not providing efficiency of this system.\cite{6266137}
 
 
\section*{Project Specification}
\label{specification}
Having reviewed the literature and determined what is feasible, a background study was made on the design of the project. A simple attendace tracking and management system was initially considered but given that there are already solutions to this, a new perspective of the solution to the problem found as stated in the Introduction\textsuperscript{\ref{problem}} and built. Thus, providing a solution to the problem was still worthwile and able to fill the gap found in the literature. The solution is therefore a combination of some of the methods used in the works cited as well as a further implementation of some concepts. It was decided to add a remote access to this functionality based on a discussion found in \citeauthor{Rachna}'s work. Notwithstanding, an automation loophole would exist if a lecturer is late to the lecture room because he would not be able to account for students already present. A desktop application that is utilised on the room computer was considered but after some research, the possibility of using a web application and monitoring attendance remotely was discovered. The foundation of this study is presented in the Background.

\chapter{Background}
\newacronym{SQL}{$SQL$}{Structured Query Language}
\glsadd{SQL}
\newacronym{UART}{$UART$}{Universal Asynchronous Receiver-Transmitter}
\glsadd{UART}
\newacronym{URL}{$URL$}{Uniform Resource Locator}
\newacronym{GPIO}{$GPIO$}{General Purpose Input Output}
\glsadd{GPIO}
\newacronym{I2C}{$I2C$}{Inter-Integrated Circuit}
\glsadd{I2C}
\newacronym{SPI}{$SPI$}{Serial Peripheral Interface}
\glsadd{SPI}
\newacronym{DOM}{$DOM$}{Document Object Model}
\glsadd{DOM}
 
 
The monitoring system is split into three different parts which includes the web application, Raspberry Pi and the cloud(Azure). These are also split into their respective parts with the web application containing the web server(node) and the website(React), the Cloud; AzureSQL server and AzureSQL database, the Raspberry Pi; flask server, the fingerprint and the \gls{RFID}. These terms are used in the Designs and Implementation section of this Literature.
 
\section{Web Application}
A website is often confused with a web page, a web application or a web server. According to MozillaWebDoc, A website is a collection of web pages which are grouped together in various ways. A web server is a computer that hosts websites and their supporting files, which are available on that computer. A web application is a software that can be accessed using a web browser.\cite{Whatisth32:online} It is designed for authorised users to interact with a particular resource or service. Authentication is often required to access these resources\label{authentication}. A website consists of static content that is accessible to the public. The website can be copies of web pages downloaded from the web server. From a web developer point of view the webpages are known as the clients while the web servers are known as the servers.
\newacronym{TCP}{$TCP$}{Transmission Control Protocol}
\glsadd{TCP}
\subsection{Communication between clients and servers}
\begin{itemize}
\item \textbf{Internet Connection}: This enables the transmission of data between clients and server
\item \textbf{TCP/IP}: These are communication protocols that define how data should be transmitted on the internet. IP handles the delivery of packets of information from source to destination while \gls{TCP} handles the management of these packets by joining them together in their right order and also asks for missing packets to be resent, this retransmission causes latency.
\item \textbf{HTTP}: This is an application layer protocol that determines how clients and servers communicate.
\end{itemize}
\newacronym{HTTP}{$HTTP$}{Hypertext Transfer Protocol}
\glsadd{HTTP}
\newacronym{API}{$API$}{Application Programming Interface}
\glsadd{API}
A website talks to the web server using an \gls{API}. It is a connection between computer programs or a set of rules that determine how programs communicate. It could also be referred to as the specification or an implementation. The specification is a document that demonstrates how to use or create a connection. Examples of \gls{API} specifications used in this project are; Tedious for the database connection to the Azure server, Express as a middleware for routing. An \gls{API} rule says that you should get a resource when you connect to a unique \gls{URL}. This resource is a chunk of data and is obtained in the form of a response, the \gls{URL} is in the form of a request. A request consists of mainly four things, the endpoint, the method, the header, the data;
\begin{itemize}
\item The endpoint: the endpoint can be explained as one end of a communication line where a function or a resource can be rendered when that endpoint is called. It is the \gls{URL} requested for, each endpoint should return a unique resource
\item The method: the type of request to send the server, There are variant types used, but the well-known types are:
\begin{center}
  \begin{tabular}{|c|   c   |}
    \hline
    \textbf{HTTP method} & function\\
    \hline
    GET & a request to retrieve resources from the server\\
    \hline
    POST & a request to create a new resource on a server\\
    & this could be stored on a database or used as a\\
    & variable for a function\\
    \hline
    PUT & a request used to update a resource on a server\\
    \hline
    DELETE & a request used to delete a resource referenced\\
    \hline
  \end{tabular}
\end{center}
\item header: a header contains information about the request and is useful to both the client and the server. It can serve as a means of authentication and to understand the contents of the body. Headers are categorised according to their contexts:
\begin{itemize}
  \item A \textbf{request header} provides information about request to the server
  \item A \textbf{response header} provides information about the response gotten from the server
\end{itemize}
\item The data/body: The data contains information required by the server. A GET request does not carry a body.
\end{itemize}
\textbf{HTTP status codes}: these codes represent the outcome of a request, they are responses. Responses are mainly grouped in five.\cite{HTTPresp7:online} An informational response is between \textit{100 - 199}. A successful response is between \textit{200 - 299}. A Redirection message \textit{300 - 399}. A client error response \textit{400 - 499} and server error response \textit{500 - 599}. The following were mostly used in the implementation of this project.
\begin{center}
\begin{tabular}{|c|  c   |}
  \hline
  \textbf{HTTP code} & meaning\\
  \hline
  200 & Ok, the request succeeded\\
  \hline
  400 & Bad request, the server cannot process\\
  &the request due to a client error\\
  \hline
  404 & Not Found, the server cannot\\
  &find the requested resource\\
  \hline
  405 & Method Not Allowed, the method\\
  &is recognised but is not allowed by the server\\
  \hline
  500 & Internal Server Error, the server\\
   &does not know how to handle a request\\
  \hline
\end{tabular}
\end{center}
 
\textbf{Document Object Model}: A \gls{DOM} is an interface that portrays a document with a logical tree. Each node or unit of this tree serves as an object that determines how components look on a webpage\cite{Introduc41:online}. ReactJS renders, updates and edits components on the \gls{DOM}.
 
\section{Raspberry Pi}
A portable computer with 40 \gls{GPIO} pins, a dedicated processor, memory, graphics driver and an operating system: raspbian OS which is similar to most linux distros.\cite{Arduinov52:online}
\gls{GPIO} is an uncommitted digital pin on an electronic circuit board which may be used as input or output or both, and can be configured by the user at runtime.\cite{Generalp11:online} The \gls{GPIO} allows it to interface with different modules. A module here is a small unit that can be integrated into a larger system but also maintained separately with no effect on the system. It is of a "plug-in" functionality. The \gls{RFID} and fingerprint are modules in this project.
To connect to azure from the Raspberry Pi, "pyodbc" driver is used. It is a python library. The driver files are installed on the Raspberry Pi. 
 
\subsection{RFID - RC522}
\subsubsection{Interfacing with the Raspberry Pi}
The \gls{RFID} card interfaces with the Raspberry Pi with \gls{SPI} communication but is also compatible with \gls{I2C} and \gls{UART} communication. \gls{SPI} is a synchronous serial communication that encourages communication over a short distance. It is mainly used in embedded systems. Serial communication has to do with sending one bit at a time sequentially and “synchronous” means that this communication is synchronised by a clock signal which is used to orchestrate the actions of the digital circuit: to determine when it is time to read the next bit/value. \gls{SPI} devices communicate in a full duplex mode with a master-slave principle normally with a single master. Full duplex means that data is transmitted back and forth simultaneously in this communication channel. The RC522 is mounted on a breadboard and connected to the raspberry Pi using jumper cables. LEDs where also mounted on this breadboard.
\textit{Note: The master is the Raspberry Pi.}
\vspace{1cm}
\begin{figure}[ht]
\centering
\includegraphics{Background/images/350px-SPI_single_slave.svg.png.png}
\caption{\citeauthor{FileSPIs48:online}, \citeyear{FileSPIs48:online}. \citetitle{FileSPIs48:online}}
\end{figure}
\vspace{1cm}
SPI has four main logic signals:
\begin{itemize}
\item SCLK: Serial clock, which accepts pulses obtained from the master
\item MOSI: Master Out Slave In, this is data transmitted from the master to slave
\item MISO: Master In Slave Out, this is data transmitted from the slave to master
\item CS/SS: Chip/Slave Select, an output from the master to notify data is being transmitted.
\end{itemize}
The RC522 module has 8 pins that interfaces with the Raspberry Pi;
\begin{itemize}
\item SDA: \gls{I2C}-bus serial data line input/output, it acts as a signal input when used for \gls{SPI} communication
\item SCK: \gls{SPI} serial clock input which has the same operation as SCLK
\item RST: an input for Reset and power-down of the module. It can turn off all the input pins and internal current sinks.
\item GND: for ground connection to the \gls{GPIO} pin of the master
\item IRQ: an interruption pin that could notify the master when an \gls{RFID} tag comes into the range of scan.
\item 3.3v/VCC: which powers the RC522 module
\item MOSI: has the same operation as MISO in \gls{SPI} communication. It receives data from master
\item MISO: has the same operation as MOSI in \gls{SPI} communication. It sends data to master(Raspberry Pi)
\end{itemize}
\subsubsection{Operations of \gls{RFID} module}
\gls{RFID} tags are classified by their frequencies, the four primary frequency ranges are:
\begin{itemize}
\item Low frequency (LF): they are frequencies from 30 to 300KHz
\item High frequency (HF): frequencies are from 3 to 30MHz, has a higher memory size and a longer range of transmission
\item Ultra high frequency (UHF): frequencies are from 300MHz to 3GHz
\item Microwave frequency (microwave): they function at 2.45GHz
\end{itemize}
\newacronym{PCD}{$PCD$}{Proximity coupling device}
\glsadd{PCD}
\newacronym{PICC}{$PICC$}{Proximity Integrated circuit card}
\glsadd{PICC}
The system consists of two main components, a transponder/tag and a transceiver/reader or a \gls{PCD} and a \gls{PICC} as defined in ISO 14443. The transceiver creates a 13.56MHz electromagnetic field that communicates with the tag.
This project uses a High Frequency passive card with Type A communication defined in ISO-14443, passive meaning, the tags only function when they acquire signals from a reader and relay an information-carrying signal back to the reader.
\subsection{Fingerprint - JM-101}
The JM-101 comes with a flash memory to store the fingerprints, it has a storage capacity of over 300 templates.
\subsubsection{Interfacing with Raspberry Pi}
The Fingerprint interfaces with the Raspberry Pi with \gls{UART} communication. \gls{UART} is a device that supports asynchronous serial communication, which is a form of serial communication that does not require a clock signal and is not constantly synchronized. Raspberry Pi supports asynchronous communication but with several fingerprints having distinct voltages. A USB to \gls{UART} Converter which supports both 3.3v and 5v was used although the JM-101 is compatible with a 3.3v pin. The TX pin goes to the RX pin and vice versa in the connection between the converter and fingerprint module. The USB to \gls{UART} Converter goes into the USB port of the Raspberry Pi while its pins are connected to the pins of the fingerprint module.
The pins used in interfacing with the \gls{UART} converter are:
\begin{itemize}
\item GND: ground connection
\item RXD: the receiver of data
\item TXD: transmits data
\item 3V3: powers the fingerprint module
\end{itemize}
\subsubsection{Operations of fingerprint module}
Fingerprint processing can either be for enrollment or matching. The user enters his finger two times when enrolling, the two templates are used to generate a template after processing their results.
A matching algorithm is used to compare fingerprint templates stored in the flash memory with the one read by the fingerprint module. There are two types of matching; 1:1 and 1:N. In 1:1 the module compares the read fingerprint with a pre-existing template that is specified in the code while \label{fingerprint 1:N}1:N checks the read fingerprint with all templates previously stored in the flash memory.
\subsection{Flask}
This creates a local web server written in python. It has a feature that binds a python function to a \gls{URL} endpoint. Another feature of flask is its ability to handle multiple requests with one function. It is in the form of a switch or if-else condition: if it is a POST it executes a codeblock but if it is a GET it executes another codeblock.
\subsection{Virtual tunnel}
Pitunnel is used to connect the Flask server with the NodeJS server that is used for development purposes. The Flask server alone runs locally and has no access to the Internet. Pitunnel is a service that allows access to any network service on the Raspberry Pi from anywhere in the world. In this project the service used is \gls{HTTP}. All that is required is to configure pitunnel to track the localhost server running and you one use the pitunnel endpoint to receive data on the Raspberry Pi. Pitunnel does this by mirroring the flask server: the root directory of the flask server becomes the root directory of the Pitunnel same as other endpoints when calling an endpoint of the flask server.
\section{Cloud - Azure}
Cloud computing has to do with the provision of computing system resources like data storage, servers, software over the internet\cite{Cloudcom53:online}. The benefits of these include reliability: the cloud provider is in charge of handling data and handles data back-up by mirroring these data at different locations, Scalability: the ability for you to "scale up" when there is an increase in demand for computing power. Scalability ensures the appropriate amount of resources for a task is provided. 
There are different types of clouds and they have unique functions;
\begin{itemize}
\item Public cloud: Public clouds  are delivered over the internet by third-party providers like Microsoft Azure, Amazon Web Services etc.
\item Private cloud: Private cloud resources are solely designed for a single organisation. These resources can be handled by a third-party or hosted by that organisation and located on their own personalised data centre.
\item Hybrid cloud: a combination of a private and public cloud and aims to give the user more flexibility. An organisation could host a private cloud with sensitive data and connect with a public cloud to host its application
\end{itemize}
\newacronym{IaaS}{$IaaS$}{Infrastructure as a service}
\glsadd{IaaS}
\newacronym{PaaS}{$PaaS$}{Platform as a service}
\glsadd{PaaS}
\newacronym{SaaS}{$SaaS$}{Software as a service}
\glsadd{SaaS}
These clouds provide their services according to models, there are mainly three according to \cite{National47:online};
\begin{figure}[h]
\centering
\includegraphics[scale=0.8]{Background/images/Cloud_computing_service_models_(1).png}
\caption{\citeauthor{FileClou30:online}, \citeyear{FileClou30:online}. \citetitle{FileClou30:online}}
\end{figure}
\begin{itemize}
  \item \gls{IaaS}: this provides the user with resources like virtual machines, storage, operating systems
  \item \gls{PaaS}: provides the user with an environment to develop software without management of the underlying resources required like servers, operating systems
  \item \gls{SaaS}: the cloud provider supply users with software over the Internet, this is mostly done by connection to a web browser, here the consumer does not management the underlying resources
\end{itemize}
This project uses a public cloud type provided by Microsoft Azure and \gls{IaaS} for the storage and \gls{PaaS} for the server. AzureSQL server hosts the SQL database and handles connection to other devices, programs or services.
My options were NoSQL("Not Only SQL") or SQL also known as Sequel.
SQL is a programming language used to manage data in a relational database. T-SQL was specifically used for querying data in Microsoft SQL Server.




\subsection{Terms Used}[h]
A lot of these terms will be used to talk about managing the database;
\begin{itemize}
\item Table: Tables contain all data in database, they do this in the form of a row and column format, where each row is a unique record and each column a unique attribute
\item Rows: represents a collection of attributes that make up a data record
\item Columns: Columns represent different attributes of a table. They can also be called fields or attributes. a single column has a similar data type for all rows.
\item Primary key: A primary key is a column or a group of columns that acts as an identifier of each row of a table.
\item Foreign key: A foreign key can be a column or a group of columns that allows a link between two related tables by referring to the Primary Key of another table
\item Database relationships: database tables sometimes share certain relationships with each other. This relationship can be represented in mainly three ways;
 \begin{itemize}
   \item one-to-one relationship: a one-to-one is when both tables have only online record stored on each side. One record in a table relates to one and only one record in the other table and vice versa.
   \item one-to-many relationship: When the primary-key table can relate to zero to many records in another table.
   \item many-to-many relationship: each records in both tables can relate to none and above in the other table.\label{manytomany} Most relational database systems do not support this relationship and a third table is used to link the both tables to each other.
\end{itemize}
\end{itemize}
 
 
 
\subsection{Communication between the Database and other devices}
The azure web portal provides a list of connection strings to interact with the sql database. For this project ODBC driver was used with SQL authentication connection string:
\begin{verbatim}
Driver={ODBC Driver 13 for SQL Server};
Server=tcp:csproj.database.windows.net,1433;
Database=vantracker;
Uid=g3ar;
Pwd={your_password_here};
Encrypt=yes;
TrustServerCertificate=no;
Connection Timeout=30;
\end{verbatim}
 
 The Design \& Implementation explain the choices taken for each part of the system and why they were chosen.
 
 


\chapter{Design and Implementation}
While working on this project, I used the agile development, why> because the design and implementations of embedded devices such as the fingerprint and RFID is uncertain.  initially I wanted to use a desktop application because 

during my research I got inquiries from MozillaWebDoc

API


Why ReactJS?

Why NodeJS?
It's really good at handling simultaneous connections. In this project we're sending and receiving data from the database, the flask server and the front-end

Why python?
python has a library for the fingerprint module(Mega 2560 R3) I used. It also had a library for the raspberry pi(RC522) and how easy it is to write and test code.

Interfaces?
USB-UART Converter 6-in-1 research

\section{Approach}

I decided to go with Full Stack development. why? to understand and solidify my background in the fundamentals of web development. 

I divided this system into components 
I went with a website because it's easy to access from anywhere that has an internet connection and each part of the system can access each other cause they are all connected to the internet. 

Single Web Applicaton

I first thought about the languages and frameworks I would like to use. For the website (React), the webserver (Node), and RFID and Fingerprint (Python). After that, I took into account the specification of the project; the software design process, 

At first I wanted to call each python scripts from the webserver but I ran into a series of errors coming from Raspberry pi not being able to grant access for that, and then I found out how to use flask and I came up with the idea of calling these scripts from a http endpoint of the flask server.
\chapter{Results}
\section{Actions}
\section{Real World Testing}
\chapter{Conclusions}
An evaluation of the project

\section{Discussion and Future Improvements}

Overall, this project achieved its aim of implementing an Attendance monitoring system that (does what it does). However, the uniqueness of this system can be extended.

Firstly by adding/integrating more tracking methods for different scenarios efficiency will no longer be a problem.
A QR code method that uses geo-location on the mobile application, that checks location before scanning QR code.

As much as RFID has a simpler implementation, it comes with a lot of limitations. It's easy to fool the system. In the near-future having a virtual NFC card just like bank cards in phones will be way more safer with respect to the system.

The timer on the website alongside with the allocated time from the database will be used to validate how long the class was held along other things.

% \include{chapter6/ch6}



%% APPENDICES %%
\appendix
\chapter{First Appendix}
\vspace*{-0.3in}
\begin{figure}[hb!]
\begin{center}
\makebox[\textwidth][c]{\includegraphics[width=1\textwidth]{appendix/images/example_graph.png}}
\end{center}
\caption{Again, the example graph.}
\end{figure}


%\include{appendix2/appendix2}

%% BACKMATTER %%
% The pages inside of backmatter are in Arabic numerals and the chapters will not have numeration
\backmatter

% BIBLIOGRAPHY WITH BIBTEX %
%*******************************************************
% Bibliography
%*******************************************************
\cleardoublepage
\phantomsection
\addcontentsline{toc}{part}{References}
\printbibliography[title=References]
\nocite{*}
\vspace{2.5cm}
\begin{Large}Websites consulted\end{Large}
\begin{itemize}
\item ISO -- \url{https://www.iso.org/}
\item NIST -- \url{https://www.nist.gov/}
\item MozillaWebDoc -- \url{https://developer.mozilla.org/en-US/s}
\end{itemize}


% All the sources are described in a file named bibliography.bib
% if you want to cite one in the text:
% \citep{label}
% In order to update the bibliography you have to execute:
% bibtex main (without ".tex")

% INDEX %
\cleardoublepage
% To help hyperref to jump to the correct page
\phantomsection
% To add the Index in the table of contents
\addcontentsline{toc}{chapter}{Index}
% Prints the Index
\printindex
% To add an item in it, write the \index{WORD} after the word to highlight:
% WORD\index{WORD}
% In order to update the Index you have to execute:
% makeindex main (without ".tex")

% A typical session involving a bibliography, an index and so on would require:
% pdflatex main
% makeindex -s main.ist -t main.alg -o main.acr main.acn
% makeindex -s main.ist -t main.glg -o main.gls main.glo
% bibtex main
% pdflatex main
% pdflatex main
% makeindex main
% makeindex -s main.ist -t main.alg -o main.acr main.acn
% makeindex -s main.ist -t main.glg -o main.gls main.glo
% pdflatex main
% pdflatex main
\end{document}