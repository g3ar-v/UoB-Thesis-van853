\providecommand\phantomsection{} \phantomsection
%\addcontentsline{toc}{part}{Abstract}
\begin{center}
%\hrule
\providecommand\pdfbookmark[3][]{} \pdfbookmark[0]{Abstract}{bm:Abstract}
\vspace*{1in}
\textbf{ABSTRACT}\\[2\baselineskip]
% \vspace*{.1in}
\end{center}

The shortcoming in the attendance monitoring system of the University of Birmingham observed over the past four years has raised some concerns. The shortcoming is evident in tracking attendance in a Wide Area Network due to high numbers of users and range (even with boosters) of connectivity. The University attendance is usually answered on canvas in a quiz form. However, most times users are unable to access the internet as well as to track when out of data on their mobile phones. This prevents the authority from tracking lecture activities in the University. Areas abound where there is absence of reliable internet and this issues arise with tracking attendance in monitoring systems.
The aim of my work is to develop an attendance system with a raspberry pi, a website, RFID and a fingerprint module with Azure cloud computing service where a database and internet exception handler is designed and implemented to store attendance data temporarily on the raspberry pi. This would later be sent when there is internet access. 
This produces a system that is capable of tracking attendance in case of short internet latency gaps given different options and tracking choices. It seems to curb the problem of not connecting to the database by storing data temporarily on the raspbery-pi  and sending it back up when the internet is on. The University of Birmingham could adapt this system to solve its monitoring lapses
\textbf{}



