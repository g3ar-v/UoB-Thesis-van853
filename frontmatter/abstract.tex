\providecommand\phantomsection{} \phantomsection
\newacronym{RFID}{$RFID$}{Radio-Frequency Identification}
\glsadd{RFID}
%\addcontentsline{toc}{part}{Abstract}
\begin{center}
%\hrule
\providecommand\pdfbookmark[3][]{} \pdfbookmark[0]{Abstract}{bm:Abstract}
\vspace*{1in}
\textbf{ABSTRACT}\\[2\baselineskip]
% \vspace*{.1in}
\end{center}

The shortcoming in the attendance monitoring system of the University of Birmingham observed over the past four years has raised some concerns. The shortcoming is evident in tracking attendance in a Wide Area Network due to high numbers of users and range of connectivity even with boosters. The University attendance is usually answered on canvas in a quiz form. However, most times users are unable to access the internet as well as to track when out of data on their mobile phones. This prevents the authority from tracking lecture activities in the University. There are areas where due to absence of reliable internet, issues arise with tracking attendance in monitoring systems.
This work aimed to develop an attendance system with a Raspberry pi, a web application, \gls{RFID} and a fingerprint module with Azure cloud computing service. It also aimed to design and implement a database and internet exception handler to store attendance data temporarily on the Raspberry pi. This would later be sent when there is internet access.
This produces a system that is capable of tracking attendance in case of short internet latency gaps given different options and tracking choices. Notwithstanding a few challenges, the aim of the project was achieved. The solution provided seems to curb the problem of not connecting to the database by storing data temporarily on the Raspberry pi and sending it back up when the internet is on. Organisations and Educational institutions could adapt this system to solve the identified lapses in their monitoring systems.
