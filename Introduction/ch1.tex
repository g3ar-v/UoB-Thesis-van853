\newacronym{IoT}{$IoT$}{Internet of Things}
\glsadd{IoT}

\chapter{Introduction} 
The moment mankind found the need to keep track of attendance in different environments, automation, accuracy and reliability have been the principle. According to \citeauthor{HistoryO76:online} \& \citeauthor{TheEvolu41:online}, the use of attendance tracking dates back to the early 1800s during the Industrial Revolution when families moved from rural areas into the cities for factory labour. Due to the unhealthy and hazardous nature of most of these jobs, the government had to set regulations to enhance the working conditions. A method to keep record of hours spent by workers was developed. The mechanical clock created by William Bundy in 1888, aided in the advancement of this system. It was used to record the time a worker entered and left the factory. This was done on time cards. Overtime, smaller electrical clocks replaced the mechanical clocks and improvements were made in the time cards. In the 1990s, a new Time software was developed to replace the mechanical clocks which were less efficient and expensive. Excel sheets were a new method of storing attendance data. Absence management was monitored effectively.\cite{HistoryO76:online, TheEvolu41:online} Monitoring of attendance in organisations has enhanced the satisfaction of both the employer and the employees due to the transparency of the monitoring systems as a tangible contract held by both parties. In educational institutions, these systems assess the level of participation of users, aided in the procurement  of valid research data and the management of attendance records. Though there are Attendance Monitoring Systems with the capacity to efficiently do their jobs, there are places where lack of a reliable internet connectivity as a result of many external factors like latency and packet loss impede the efficiency of these systems to function effectively. This project investigates through the field of \gls{IoT} to find a distinct solution that solves this problem efficiently.
 
\section{Project Aims}
 
The aim of this project is to build an attendance monitoring system that is able to track attendance with a diverse number of methods and to render a novel solution to the argument "Inefficiency in attendance monitoring systems”\label{problem}. The focus will be on integrating this system to use an \gls{RFID} system, a manual website method and a fingerprint for tracking attendance while also monitoring attendance with a short lapse in internet or database connectivity. The website aids an administrator or lecturer in interacting with the system. The evaluation criterias of this system will be based on its ability;
\begin{itemize}
 \item to track attendance with any of the referred methods,
 \item to handle the exceptions of the internet or database connection problem.
 \item to show results based on each attendance class
\end{itemize}
This project also aims to give the administrator remote access to the \gls{RFID} and fingerprint systems from anywhere in the world theoretically based on its design and implementation. The results will be determined by the use of a dummy dataset. \gls{RFID} tags will be enrolled with these datasets and fingerprints of certain individuals.
\section{Related Works}
Most of the literature cited, agree that manual attendance is time-consuming, prone to mistakes and insecure. Nonetheless, a few of the literatures have revealed measures used to solve the problem of inefficiency in attendance monitoring systems. Papers exist which narrate how \gls{RFID} has been used as a solution. \citeauthor{KaziARP} considers it a waste of time monitoring both body temperature and attendance in schools which was a problem caused by the COVID-19 pandemic. They preferred to join these two systems to monitor attendance using \gls{RFID} as well as monitor body status and store the data to be accessed remotely across the globe.\cite{KaziARP} This influenced the merging of multiple attendance methods in this work and also the use of remote access to the website. \citeauthor{Tt2021} proposed a smart attendance system that is appropriate to evade the use of manual attendance system and reduce manual work of teachers and administrators. By building a system that monitors attendance with passive \gls{RFID} and sends the data attained to the Cloud. Futhermore, a GPS system is also used to get the live location of the student.\cite{Tt2021} \citeauthor{Www2012} seek to create a fast, automated, reliable and accurate system that monitors attendance by reading tags in the lecture room using an Ultra-High frequency \gls{RFID} reader and a middleware to sort, convert and relay data to a database server or to the appropriate system.\cite{Www2012} \citeauthor{Bhagat2020} on the other hand, proposes a system that effectively and efficiently monitors attendance with a website and \gls{RFID} reader using an MQTT protocol and a NodeMCU firmware with the aim of eradicating manual attendance and saving cost.\cite{Bhagat2020}
Some of these works also present solution that uses fingerprint as a method of tracking.
\citeauthor{soe2018implementation} proposes fingerprint based biometric attendance system that monitors attendance and prevents proxies(when someone scans attendance for someone else). The data is stored in an SD card and later extracted. The literature also emphasizes the reliability and security of the system to prohibit truancy and develops a solution to stop it. As a tracking method, fingerprint is more reliable, accurate and cost effective based on the fact that every individual has a unique fingerprint. Furthermore, a GSM system is put in place to alert guardians, if a student does not attend. One limitation of this solution is that it is not automated. A teacher or an authorised personnel has to get an SD card from the microcontroller to store attendance in the database.\cite{soe2018implementation} In another study, \citeauthor{Rachna} sought to reduce the entire time taken to capture attendance of students in a class by designing a portable device that is circulated around the classroom for students to mark their attendance. The device is a cloud-based system that relies on a fingerprint scanner, web application to track attendance, it also reports absentees to their guardians using SMS and a mobile application for students to see their attendance record. The fingerprint module uses MQTT as a means of communicating with its Web application. It seems that the author aims to solve attendance by proxy since only people in the room can sign in as present. Although this solves the problem of proxy, it impaires automation as circulating this device among a large number of students might not seem efficient. The authors future remarks hinted at a device that operates in an offline mode which stores the attendance locally when there is no network connectivity and transfers it to the cloud when there is a connection.\cite{Rachna} However, having perused some literatures in this field, this is the only paper that addresses the implementation but failed to implement. It is therefore the intention of this project to implement a system which stores a temporary file directory on the Raspberry pi before being sent to the database. Similarly \citeauthor{Rahman2021}, aims to provide a reliable method of monitoring attendance with the use of a fingerprint automated system. This is with the use of a user interface to interact with both the lecturers and student. The user interface acts as a response system for students but can be used to start attendance with the use of a lecturer's fingerprint.\cite{Rahman2021} Lessons from \citeauthor{Rahman2021}'s work be useful in this work.

There are other methods used to monitor attendance. \citeauthor{9716261} proposes an attendance system using Speech recognition with the aim of impeding the use of manual attendance and increasing efficiency by scanning attendance with the use of voice input on a mobile application. A speech recognizer converts the audio input into a valid message which matches a database record.\cite{9716261}
 
Given the existing projects and literature in this field, most of the authors reviewed, focused on upgrading existing methods to solve the inefficiency of the attendance monitoring system.\cite{KaziARP, Tt2021, Www2012, Bhagat2020, soe2018implementation, Rachna} In as much as efficiency is the central theme in these literatures, it has been noted that solutions based on only one perspective have been applied by these authors. Notwithstanding, efficiency has always been the driving force of these efforts. My argument is, "why not use multiple methods for different scenarios"?, given that in some cases some methods might be more efficient in taking attendance and having them integrated makes the system flexible and better or even more efficient(in theory) in tracking attendance. A paper exists that implements this. \citeauthor{6266137} proposed a system that uses NFC and specifically a fingerprint module as a means to curb truancy, monitor attendance and provide reliable information on student attendance from a component called the back office. However, the central theme of this paper has to do with monitoring attendance in general and not providing efficiency of this system.\cite{6266137}
 
 
\section*{Project Specification}
\label{specification}
Having reviewed the literature and determined what is feasible, a background study was made on the design of the project. A simple attendace tracking and management system was initially considered but given that there are already solutions to this, a new perspective of the solution to the problem found as stated in the Introduction\textsuperscript{\ref{problem}} and built. Thus, providing a solution to the problem was still worthwile and able to fill the gap found in the literature. The solution is therefore a combination of some of the methods used in the works cited as well as a further implementation of some concepts. It was decided to add a remote access to this functionality based on a discussion found in \citeauthor{Rachna}'s work. Notwithstanding, an automation loophole would exist if a lecturer is late to the lecture room because he would not be able to account for students already present. A desktop application that is utilised on the room computer was considered but after some research, the possibility of using a web application and monitoring attendance remotely was discovered. The foundation of this study is presented in the Background.
