\chapter{Introduction}
Right from the moment we(humans) discovered the need to keep track of attendance in different environments, automation, accuracy and reliability have been the principle. In as much as we have Attendance Monitoring Systems which efficiently do their jobs,there are places where there are unreliable internet connectivity, caused by many external factors like latency, packet loss. This project investigates through the field of IoT to find a solution that solves this problem.


\section{Project Aims}

The aim of this project is to build a attendance monitoring system that is able to track attendance with a diverse number of methods while also being able to this with unreliable internet connectivity. 

The main evaluation criteria
\newacronym{IoT}{$IoT$}{Internet of Things}
\glsadd{IoT}

\newacronym{RFID}{$RFID$}{Radio-Frequency Identification}
\glsadd{RFID}

\section{Related Works} 
These are ways most authors have used to solve the issue of inefficiency in attendance monitoring systems.
\subsection{General texts}
The authors think it is a waste of time monitoring both body temperature and attendance in schools which is a problem caused by the COVID-19 pandemic, and seeks to join these two systems to monitor attendance as well as monitor body status and store the data to be accessed remotely across the globe. They use This influenced the merging of multiple attendance methods in my work.
The author proposed a smart attendance to evade the use of manual attendance system and reduce manual work of teachers and administrators, by building a system that monitors attendance with passive RFID and sends the data attained to the Cloud.
The authors seek to create a fast, automated, reliable and accurate system that monitors attendance by reading tags in the lecture room using an Ultra-High frequency reader and a middleware to sort, convert and relay data to the database system. 
The author proposes a system that effectively and efficiently monitors attendance monitors with a website and RFID reader using an MQTT protocol and a NodeMCU firmware with the aim of eradicating manual attendance saving cost.


\subsubsection{Theoretical Approaches}
With the currently existing projects and literature in this field, a lot of the authors centralize on the inefficiency of attendance monitoring system based on the previous methods used and devise new methods as a solution[1,2 and more]. 

\subsubsection{Empirical Research}
[2] talks about the reliability and security of a system to prohibit truancy and develops a solution to stop it, Fingerprint is used as a tracking method because it's more reliable, accurate and cost effective (explain) and a GSM to alert guardians, if a student doesn't attend.

A limitation of this solution, it's not automated, teacher or an authorized has to get SD card from micro controller to store attendance in database.


\subsection{Central texts}

In as much as efficiency is quite the central theme in these literatures, I see only one solution being applied in these literatures and that has to do with using a more efficient method, why not use multiple methods for different scenarios, in some cases some methods might be more efficient in taking attendance and having them integrated makes the system flexible and better or even more efficient(in theory) in tracking attendance.

In "An Investigation on the viability of using IoT for student survey and attendance monitoring"[4]



A solution of mine was gotten from [3],which stores a temporary file directory before being sent to the database.
